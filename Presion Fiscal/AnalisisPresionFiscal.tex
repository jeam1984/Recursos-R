% Options for packages loaded elsewhere
\PassOptionsToPackage{unicode}{hyperref}
\PassOptionsToPackage{hyphens}{url}
%
\documentclass[
]{article}
\usepackage{lmodern}
\usepackage{amssymb,amsmath}
\usepackage{ifxetex,ifluatex}
\ifnum 0\ifxetex 1\fi\ifluatex 1\fi=0 % if pdftex
  \usepackage[T1]{fontenc}
  \usepackage[utf8]{inputenc}
  \usepackage{textcomp} % provide euro and other symbols
\else % if luatex or xetex
  \usepackage{unicode-math}
  \defaultfontfeatures{Scale=MatchLowercase}
  \defaultfontfeatures[\rmfamily]{Ligatures=TeX,Scale=1}
\fi
% Use upquote if available, for straight quotes in verbatim environments
\IfFileExists{upquote.sty}{\usepackage{upquote}}{}
\IfFileExists{microtype.sty}{% use microtype if available
  \usepackage[]{microtype}
  \UseMicrotypeSet[protrusion]{basicmath} % disable protrusion for tt fonts
}{}
\makeatletter
\@ifundefined{KOMAClassName}{% if non-KOMA class
  \IfFileExists{parskip.sty}{%
    \usepackage{parskip}
  }{% else
    \setlength{\parindent}{0pt}
    \setlength{\parskip}{6pt plus 2pt minus 1pt}}
}{% if KOMA class
  \KOMAoptions{parskip=half}}
\makeatother
\usepackage{xcolor}
\IfFileExists{xurl.sty}{\usepackage{xurl}}{} % add URL line breaks if available
\IfFileExists{bookmark.sty}{\usepackage{bookmark}}{\usepackage{hyperref}}
\hypersetup{
  pdftitle={Ejemplo de periodismo de datos con R},
  pdfauthor={Isaac Gonzalez - www.datascience4business.com},
  hidelinks,
  pdfcreator={LaTeX via pandoc}}
\urlstyle{same} % disable monospaced font for URLs
\usepackage[margin=1in]{geometry}
\usepackage{graphicx,grffile}
\makeatletter
\def\maxwidth{\ifdim\Gin@nat@width>\linewidth\linewidth\else\Gin@nat@width\fi}
\def\maxheight{\ifdim\Gin@nat@height>\textheight\textheight\else\Gin@nat@height\fi}
\makeatother
% Scale images if necessary, so that they will not overflow the page
% margins by default, and it is still possible to overwrite the defaults
% using explicit options in \includegraphics[width, height, ...]{}
\setkeys{Gin}{width=\maxwidth,height=\maxheight,keepaspectratio}
% Set default figure placement to htbp
\makeatletter
\def\fps@figure{htbp}
\makeatother
\setlength{\emergencystretch}{3em} % prevent overfull lines
\providecommand{\tightlist}{%
  \setlength{\itemsep}{0pt}\setlength{\parskip}{0pt}}
\setcounter{secnumdepth}{-\maxdimen} % remove section numbering

\title{Ejemplo de periodismo de datos con R}
\author{Isaac Gonzalez - www.datascience4business.com}
\date{}

\begin{document}
\maketitle

\hypertarget{preparacion-del-analisis}{%
\section{1. PREPARACION DEL ANALISIS}\label{preparacion-del-analisis}}

El objetivo es aportar información rigurosa sobre el debate en cuanto al
nivel impositivo de España:

\begin{itemize}
\tightlist
\item
  Analizando la presión fiscal (impuestos recaudados sobre el PIB)
\item
  Analizando el esfuerzo fiscal (nivel de esfuerzo del contribuyente
  teniendo en cuenta su poder adquisitivo)
\item
  En qué se invierte el dinero recaudado
\end{itemize}

\hypertarget{analisis-de-la-presion-y-esfuerzo-fiscal}{%
\section{2. ANALISIS DE LA PRESION Y ESFUERZO
FISCAL}\label{analisis-de-la-presion-y-esfuerzo-fiscal}}

Fuente de los datos:
\url{https://www.elblogsalmon.com/indicadores-y-estadisticas/impuestos-esfuerzo-fiscal-espanoles-desmedido}

Se ha utilizado tecnología OCR sobre la fuente citada para extraer los
datos de la imagen y poder utilizarlos en los análisis.

Tras el proceso de extracción con OCR y una primera limpieza de datos
hemos obtenido la siguiente tabla:

\begin{verbatim}
##            pais presion_valor presion_ranking esfuerzo_valor esfuerzo_ranking
## 1        Grecia        0.4107              11        26.2760                1
## 2      Portugal        0.3653              13        20.2730                2
## 3        Italia        0.4291               9        16.1710                3
## 4       Francia        0.4707               1        13.9460                5
## 5        España        0.3574              15        15.1370                4
## 6       Bélgica        0.4621               2        12.0280                6
## 7       Austria        0.4384               4        10.6580                Y
## 8     Finlandia        0.4249               8        10.6450                8
## 9      Alemania        0.4190               o        10.4950                9
## 10       Suecia        0.4290               6        10.0000               11
## 11    Dinamarca        0.4546               3         0.8856               15
## 12   ReinoUnido        0.8850              16        10.3060               10
## 13   Luxemburgo        0.4149              10         0.8947               14
## 14 Países_Bajos        0.3815              12         0.9074               13
## 15        Japón        0.3377              17         0.9149                a
## 16      Noruega        0.4281               7         0.7376               16
## 17     Islandia        0.3594              14         0.6629               17
## 18      Irlanda        0.2186              20         0.6051               18
## 19        EEUU.        0.2505              19         0.4673               19
## 20        Suiza        0.2881              18         0.4295               20
\end{verbatim}

Vemos que en los rankings hay datos que el ocr no ha podido extraerlos
bien.

Pero los valores sí parecen estar todos bien.

Así que vamos a generar de nuevo los rankings a partir de los valores.

Nos fijamos en que el valor 1 se le da al valor más alto en ambos casos.

\begin{verbatim}
##            pais presion_valor presion_ranking esfuerzo_valor esfuerzo_ranking
## 1        Grecia        0.4107              12        26.2760                1
## 2      Portugal        0.3653              14        20.2730                2
## 3        Italia        0.4291               6        16.1710                3
## 4       Francia        0.4707               2        13.9460                5
## 5        España        0.3574              16        15.1370                4
## 6       Bélgica        0.4621               3        12.0280                6
## 7       Austria        0.4384               5        10.6580                7
## 8     Finlandia        0.4249               9        10.6450                8
## 9      Alemania        0.4190              10        10.4950                9
## 10       Suecia        0.4290               7        10.0000               11
## 11    Dinamarca        0.4546               4         0.8856               15
## 12   ReinoUnido        0.8850               1        10.3060               10
## 13   Luxemburgo        0.4149              11         0.8947               14
## 14 Países_Bajos        0.3815              13         0.9074               13
## 15        Japón        0.3377              17         0.9149               12
## 16      Noruega        0.4281               8         0.7376               16
## 17     Islandia        0.3594              15         0.6629               17
## 18      Irlanda        0.2186              20         0.6051               18
## 19        EEUU.        0.2505              19         0.4673               19
## 20        Suiza        0.2881              18         0.4295               20
\end{verbatim}

\pagebreak

Ahora ya pasamos a la fase de visualización.

Analizamos primero como está España en cuanto a presión fiscal
(impuestos recaudados / PIB)

\includegraphics{AnalisisPresionFiscal_files/figure-latex/unnamed-chunk-7-1.pdf}

\pagebreak

Pero ahora vemos el esfuerzo fiscal (ajuste por renta per capita)

\includegraphics{AnalisisPresionFiscal_files/figure-latex/unnamed-chunk-8-1.pdf}

\pagebreak

Y ahora vamos a crear un mapa de posicionamiento con los 2 indicadores a
la vez

\includegraphics{AnalisisPresionFiscal_files/figure-latex/unnamed-chunk-9-1.pdf}

\pagebreak

\hypertarget{analisis-de-las-partidas-del-gasto}{%
\section{3. ANALISIS DE LAS PARTIDAS DEL
GASTO}\label{analisis-de-las-partidas-del-gasto}}

Parece que aunque a nivel del indicador de presión fiscal España todavía
tendría recorrido para subir impuestos, cuando analizamos el esfuerzo
real que los impuestos actuales ya suponen para el contribuyente solo
hay 3 países por encima de España.

En muchos casos se justifica ese esfuerzo alegando que es necesario para
mantener servicios como la sanidad, la educación o la protección social.

Vamos a visualizar cómo se distribuyen las partidas del gasto público
según datos del Ministerio de Hacienda obtenidos de:

\url{https://www.epdata.es/datos/gasto-publico-espana-datos-graficos/256}

Gráfico: Gasto público en España por partida de gasto

A partir de la fuente citada y limpiando y organizando los datos
obtenemos la siguiente tabla:

\begin{verbatim}
## # A tibble: 10 x 6
##    ano   partida                            gasto  total  porc porc_acum
##    <chr> <chr>                              <dbl>  <dbl> <dbl>     <dbl>
##  1 2018  proteccion_social                 203116 501497 40.5       40.5
##  2 2018  salud                              72017 501497 14.4       54.9
##  3 2018  servicios_publicos_generales       67705 501497 13.5       68.4
##  4 2018  asuntos_economicos                 48909 501497  9.75      78.1
##  5 2018  educacion                          48095 501497  9.59      87.7
##  6 2018  orden_publico_y_seguridad          21997 501497  4.39      92.1
##  7 2018  ocio_cultura_y_religion            13353 501497  2.66      94.8
##  8 2018  proteccion_del_medio_ambiente      10604 501497  2.11      96.9
##  9 2018  defensa                            10283 501497  2.05      98.9
## 10 2018  vivienda_y_servicios_comunitarios   5418 501497  1.08     100
\end{verbatim}

\pagebreak

Que podemos ver de forma gráfica

\includegraphics{AnalisisPresionFiscal_files/figure-latex/unnamed-chunk-13-1.pdf}

\hypertarget{conclusiones}{%
\section{4. CONCLUSIONES}\label{conclusiones}}

\begin{itemize}
\item
  La presión fiscal en España es de las más bajas, ocupando el puesto 16
  en un ranking de 20 países
\item
  Sin embargo, el esfuerzo real para los contribuyentes es de los
  mayores, ocupando el puesto 4 en un ranking de 20 países
\item
  El gasto de sanidad + educación suma el 23.95 por ciento
\item
  Y si añadimos protección social suma el 64.45 por ciento
\item
  QUE CADA UNO SAQUE SUS PROPIAS CONCLUSIONES!!!
\end{itemize}

\end{document}
